\section{Exploration des données} 
    \subsection{Données}
    \subparagraph{} Les données fournies sont des cotations de spreads de CDS pour les maturités
    allant de 1ans à 10ans, entre 16/08/2005 et
    30/09/2010. 
    Cette periode a ete marque par le debut de la crise financiere en 2008, et en
    particulier par la faillite de la filiale newyorkaise de la societe AIG le 16
    septembre 2008. 
    \subparagraph{}
    Le marche des CDS est consideree comme une des riaisons principales de la chute
    du groupe Americain AIG.En effet, Le march des CDS est passe de 6 396 milliards de dollars
    amricains à fin 2004 à 57 894 milliards à fin 2007, prenant le caractère d'une
    bulle financière.


    j'ai choisi ici de faire mon étude pour les spreads de 1ans car l'évolution des
    spreads est presque identiques pour les différentes maturités.

    \begin{verbatim}
    > colnames(data)
    [1] "cot" "S1"  "S2"  "S3"  "S4"  "S5"  "S6"  "S7"  "S8"
    "S9"  "S10"
    > head(data$cot)
    [1] "2005-08-16" "2005-08-17" "2005-08-18" "2005-08-19" "2005-08-22"
    "2005-08-23"
    > head(data$S1)
    [1] 1.1 1.1 1.1 1.1 1.1 1.1
    \end{verbatim}

    \subsection{Chronogramme} 
    Notre première étape est donc de voir l'allure de la courbe :
    \begin{figure}[H]
        \centering
        \label{fig:chrono} 
        \includegraphics[width=4in,heigth=4in,angle=270]{chrono} 
        \caption{\it Le chronogramme des spreads de maturité 1 ans } 
    \end{figure}

    On remarque bien que la variance de la courbe augmente avec le temps. Cela
    montre de premier abord que le processus n'est pas stationnaire, autrement
    on aurait eu une variation stable au long du processus.

    On constate aussi une croissance importante du processus, et cela favorise bien
    l'existence d'une tendance croissante.

    L'existence d'une saisonnalité est moins évidente  à détecter. 

    \subsection{Stabilisation la variance}
    L'augmentation de la variance au fil du temps peu être diminue grâce a
    certaines transformation. Nous allons choisir celle  de Box-Cox définie par :
    \[
        f_\lambda(X_t)=
        \begin{cases}
            \lambda^{-1}(X_t^\lambda - 1), & X_t \geq 0,\lambda \ge 0, \\
            \ln X_t,                       & X_t \ge  0,\lambda  =   0,
        \end{cases}
    \]
    pour cela comparons les transformations pour les valeurs de
    $\lambda=0,0.5,1.5$
    \begin{figure}[H]
        \centering 
        \label{fig:Boxcox} 
        \includegraphics[width=4in,heigth=4in,angle=270]{boxcox} 
        \caption{\it Trace des spreads de 1 ans en appliquant les
        transformation de Box-Cox pour 3 valeurs de $\lambda$ } 
    \end{figure} 
    On remarque que la transformation logarithmique est bien adaptée  aux
    données car elle diminue l'effet de la croissance de la variance.\\
    Ainsi dans la suite on choisira de travailler sur cette transformation
    $\ln(X_t)$


\begin{verbatim}
S1 <- log(data$S1)
\end{verbatim}

    \subsection{Lag plot}
    Un lag plot ou diagramme retardé est le diagramme de dispersion 
    des points ayant pour abscisse la série retardée de k instants 
    et pour ordonnée la série non retardée.\\ 
    Si le diagramme retardé suggère une corrélation entre les deux séries, 
    on dit que la série présente une auto-corrélation d’ordre k.

    Les $X_i$ représentes les spreads de maturité "i" années.
    \begin{figure}[H] 
        \begin{center} 
            \includegraphics[height=4in, width=4in,angle=270]{lagplot} 
        \end{center} 
        \caption{\it le lag plot des données de $X_{t-1}$ jusqu'à $X_{t-56}$} 
        \label{fig:lagplot} 
    \end{figure}

    On constate graphiquement que la fonction d'auto-corrélation du processus est décroissante. On peut donc affirmer
    que le processus est faiblement stationnaire.

    On constate donc que la fonction fonction d'auto-corrélation s'annule presque quand h augmente
    . Cela nous favorise l'hypothèse que le processus est k-depandant. \\
    le phénomène de périodicité est peu favorisé ici car sinon on aura pas une décroissance parfaite de 
    la fonction d'auto-corrélation.

    le processus semble ne pas avoir de saisonnalité particulière.
    Le processus ne peut pas être AR ou MA car sa fonction d' auto-corrélation ne s'annule pas :
    \begin{figure}[H]
        \begin{center}
            \includegraphics[width=4in,height=4in,angle=0]{ac} 
        \end{center}
        \caption{}
        \label{fig:cor}
    \end{figure}

    \subsection{Étude de la saisonnalité et de la tendance}

    Dans ce paragraphe nous allons voire si on peut détecter une saisonnalité.
    Commençons alors par tracer le périodogramme :
    \begin{figure}[H]
        \centering 
        \label{fig:period} 
        \includegraphics[width=2in,heigth=2in,angle=270]{period} 
        \caption{\it périodogramme de la période 16/07/2007 - 03/12/2007 } 
    \end{figure} 

    Le périodogramme ne nous informe pas sur une probable saisonnalité.
    \subsubsection{Agregation annuelle}
    Afin de voire le comportement annuelle de notre serie nous allons agreger
    les spreads. 
    \begin{figure}[H]
        \centering 
        \label{fig:annual} 
        \includegraphics[width=4in,heigth=4in,angle=270]{Annual} 
        \caption{\it Agregation annuelle des spreads par moyenne empirique } 
    \end{figure} 

    On constate parfaitement ici le changement brusque du niveau des spreads
    apres la crise en 2008.
    \subsubsection{Agregation mensuelle}
    Pour voir le comportement des spreads mensuelle on va les agreger par
    moyenne empirique  

    \begin{figure}[H]
        \centering 
        \label{fig:meanMens} 
        \includegraphics[width=4in,heigth=4in,angle=270]{meanMens} 
        \caption{\it Moyenne par mois sur toutes les années pour les spreads
        de 1 ans } 
    \end{figure} 

    Le graphique des boxplots montre une certaine homogenite de distribution sur
     les mois, donc on en conclut que les données non pas une saisonnalité
     mensuelle particulière.
    \subsubsection{Superposition des spreads annuels} 
    Nous allons agreger les spreads par années en utilisant la moyenne
    empirique.
    \begin{figure}[H]
        \centering 
        \label{fig:polarAn} 
        \includegraphics[width=4in,heigth=4in,angle=270]{polarAn} 
        \caption{\it Representation polaire de les spreads } 
    \end{figure} 


    On constate bien que les points s'éloignent du centre d'année en
    année, ce qui traduit la tendance croissante du processus. En outre
    il n'y a pas de symétrie entre l'allure des spreads par année ce qui
    rend la présence d'une saisonnalité encore moins évident.

    Cette dernière observation nous dirige vers le sens d'aprouver une
    absence de saisonnalité.  

    La fonction \verb+decompose()+  en R permet de decomposer le processus en
    tendance et saisonnalite, alors que \verb+nsdiffs+ calcul le nombre de
    differenciation necessaires pour desaisonaliser le processus :
    \begin{verbatim}
    > decompose(S1)
    Erreur dans decompose(S1) : la srie temporelle a moins de 2 priodes
    \end{verbatim}

    \begin{verbatim}
    > nsdiffs(S1)
    Erreur dans nsdiffs(S1) : Non seasonal data
    \end{verbatim}


    \subparagraph{Conclusion}
    Sur ses resultats nous concluons que le processus ne presente aucune
    saisonnalité mais une tendance croissante.
\section{Modelisation du processus}
    \subsection{Etude des proprietes du processus}    
    \paragraph{Independance} 

    le processus est loi d'etre un processus indepandant, car certainement la
    societe n'est pas indifferente aux spreads qu'elle a emis dans le passe.
    Pour tester l'independance on choisit par exemple le test de Portemanteau
    qui est fourni par R sous \verb+Box.test()+ avec $H_0$=``le processus est
    independant''. Le test calcul
    la distance $\xi^2$ de l'auto-covariance des premiers retards. Si la p-value est tres petite ce la
    veut dire que cette auto-covariance n'est pas nulle et donc le processus :
    \begin{verbatim}
    > Box.test (S1)

        Box-Pierce test

    data:  S1
    X-squared = 1327.591, df = 1, p-value < 2.2e-16

    > Box.test (S1,  type = ``Ljung'')

        Box-Ljung test

    data:  S1
    X-squared = 1330.57, df = 1, p-value < 2.2e-16

    \end{verbatim}

    Le test retourne une p-value presque nulle donc on rejette hypothèse nulle.
    Ainsi le processus n'est pas indepandant.

    \paragraph{Stationnarite}
    Un processus stationnaire est un processus dont :
    \begin{itemize}
        \item $\forall t,E[X_t]=\mu$  
        \item $cov(X_{t+h},X_t)=\gamma(h)$ 
    \end{itemize}
    Ceci implique que le processus a une variance constante .
    Notre processus donc ne peut etre stationnaire du fait de la crise
    financière qui a eu un grand impact sur les CDS. 
    Pour tester la stationnarite de notre processus nous utiliserons les tests
    de la racine unite. 
    La fonction \verb+adf.test()+ permet de faire le test de Dickey Fuller
    où l'hypothèse nulle est la non stationnarite du processus.  
    \begin{verbatim}
    > adf.test(x = S1,alternative = ``s'')

        Augmented Dickey-Fuller Test

        data:  S1
        Dickey-Fuller = -3.0033, Lag order = 11, p-value = 0.1536
        alternative hypothesis: stationary
    \end{verbatim}

    Le test a donc une p-value non negligeable donc on ne peut pas rejeter
    l'hypothese nulle de non stationnarite 






    \subsection{Modelisation}
    Nous allons commencer par une differenciation . Notons $Y_t=\Delta X_t$
    tracons l'allure de cette difference :

    \begin{figure}[H]
        \centering 
        \label{fig:chronoD1} 
        \includegraphics[width=3in,heigth=3in,angle=270]{chronoD1} 
        \caption{\it Allure de la difference du proceccus $X_t$ } 
    \end{figure} 

    On remarque de visu que le processus est centre. Cela peut etre confirme par
    le test de Student :
    \begin{verbatim}
    > t.test(x = D1,mu = 0)

        One Sample t-test

        data:  D1
        t = 0.5449, df = 1336, p-value = 0.5859
        alternative hypothesis: true mean is not equal to 0
        95 percent confidence interval:
         -0.007258727  0.012841413
         sample estimates:
           mean of x 
           0.002791343 
    \end{verbatim}
    Le resultat du test est satisfaisant a cette egard puisque il faut accepter
    une erreur de 59\% pour pouvoir rejeter l'hypothèse nulle : le processus est
    centre.\\
    \paragraph{Test d'independance}

    \begin{verbatim}
    > Box.test(x = D1)

    Box-Pierce test

    data:  D1
    X-squared = 11.272, df = 1, p-value = 0.0007869

    > Box.test(x = D1,type = ``L'')

    Box-Ljung test

    data:  D1
    X-squared = 11.2973, df = 1, p-value = 0.0007762

    \end{verbatim}

    Le test renvoie alors une dependance du processus.
    \paragraph{Stationnarite}

    En appliquant le test de Dickey-Fuller :
    \begin{verbatim}
    > adf.test(x = D1,alternative = ``s'') # -> processus stationnaire

        Augmented Dickey-Fuller Test

        data:  D1
        Dickey-Fuller = -11.0561, Lag order = 11, p-value = 0.01
        alternative hypothesis: stationary

        Message d'avis :
        In adf.test(x = D1, alternative = ``s'') :
          p-value smaller than printed p-value
    \end{verbatim}
    On constate que le processus $Y_t$ est stationnaire.
    
    a present on peut envisager une modelisation ARMA pour le procussus
    $Y_t$.

    Examinons d'abord l'allure de l'auto-correlation de ce processus:
    \begin{figure}[H]
        \centering 
        \label{fig:D1ac} 
        \includegraphics[width=2in,heigth=2in,angle=270]{D1ac} 
        \caption{\it auto-correlation totale et partielle de $Y_t$ } 
    \end{figure} 

    on constate tout d'abord que l'ACF et la PACF s'annulent à partir du 3eme
    retard. Donc cela nous suggère immediatement trois modeles ARMA qui peuvent
    etre candidats : ARMA(0,2), ARMA(2,0) et ARMA(2,guvent etre candidats :
    ARMA(0,2), ARMA(2,0) et ARMA(2,2). 
    Comparons alors ces trois modeles en utilisant le critere AIC :

    \begin{verbatim}
    > arima(x = S1,c(0,1,2))
    Series: x 
    ARIMA(0,1,2)                    

    Coefficients:
    ma1      ma2
    -0.1269  -0.0800
    s.e.   0.0280   0.0268

    sigma^2 estimated as 0.03451:  log likelihood=353.44
    AIC=-702.88   AICc=-702.86   BIC=-687.29
    > arima(S1,c(2,1,2))
    Series: x 
    ARIMA(2,1,2)                    

    Coefficients:
    ar1     ar2     ma1      ma2
    -0.1978  0.5434  0.0524  -0.6079
    s.e.   0.1165  0.0800  0.1129   0.0791

    sigma^2 estimated as 0.03404:  log
    likelihood=362.43
    AIC=-716.86   AICc=-716.81   BIC=-690.87
    > arima(S1,c(2,1,0))
    Series: x 
    ARIMA(2,1,0)                    

    Coefficients:
    ar1      ar2
    -0.0963  -0.0526
    s.e.   0.0273   0.0273

    sigma^2 estimated as
    0.03468:  log
    likelihood=350.08
    AIC=-696.15   AICc=-696.13
    BIC=-680.56
    \end{verbatim}

    Il est clair que le Modele ARMA(2,2) est meilleur que AR(2) et MA(2)non
    seulement par le
    critere AIC mais aussi parce qu'il a la plus petite variance.



    
   

    


    
