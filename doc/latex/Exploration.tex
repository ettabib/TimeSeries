\section{Exploration du Chronogramme} 
\subsection*{Chronogramme} 
Les données sont issue de la société AIG.  
Aperçu des données :


\begin{Schunk} 
\begin{Sinput} 
> str(data)
\end{Sinput}
\begin{Soutput}
'data.frame':	1338 obs. of  11 variables:
 $ Dates.Cotations: Factor w/ 1338 levels "2005/08/16","2005/08/17",..: 1 2 3 4 5 6 7 8 9 10 ...
 $ X1             : num  1.1 1.1 1.1 1.1 1.1 1.1 1 1.1 1 0.8 ...
 $ X2             : num  0.8 0.8 0.8 0.8 0.8 0.8 0.7 0.8 0.7 0.8 ...
 $ X3             : num  1.4 1.4 1.4 1.4 1.4 1.4 1.3 1.4 1.2 1.4 ...
 $ X4             : num  3.9 3.9 3.9 3.9 3.9 3.9 3.5 3.9 3.4 3 ...
 $ X5             : num  2.2 2.2 2.2 2.2 2.2 2.2 2 2.2 1.9 2 ...
 $ X6             : num  1.6 2.1 2.5 2.1 2.1 2.1 1.9 1.7 1.4 1.5 ...
 $ X7             : num  1.7 2.8 2.8 2.8 2.8 2.8 2.6 1.8 1.5 1.7 ...
 $ X8             : num  2.2 3.3 3.2 3.4 3.4 3.4 3.2 2.3 2 2.1 ...
 $ X9             : num  3 3.6 3.6 3.8 3.8 3.8 3.6 3 2.7 2.8 ...
 $ X10            : num  3.9 3.8 4 4 4 4 3.8 4 3.7 3.8 ...
\end{Soutput}
\end{Schunk}

\begin{figure}[H]
    \centering
    \label{fig:chrono} 
    \includegraphics[width=2in,heigth=2in,angle=270]{chrono} 
    \caption{\it Le chronogramme des spreads de maturité 1 ans } 
 \end{figure} 
        
 On constate rapidement une croissance multiplicative du processus, ce qui nous impose alors une transformation 
 logarithmique
    \begin{figure}[H]
    \centering
        \label{fig:logchrono} 
         \includegraphics[width=2in,heigth=2in,angle=270]{logchrono} 
         \caption{\it transformation logarithmique des données} 
     \end{figure} 

\subsection{Lag plot}
Un lag plot ou diagramme retardé est le diagramme de dispersion 
des points ayant pour abscisse la série retardée de k instants 
et pour ordonnèe la série non retardée.\\ 
Si le diagramme retardé suggère une corrélation entre les deux séries, 
on dit que la série présente une autocorrélation d’ordre k.

Les $X_i$ représentes les spreads de maturité "i" années.
\begin{figure}[H] 
	\begin{center} 
		\includegraphics[height=4in, width=4in,angle=270]{lagplot} 
	\end{center} 
	\caption{\it le lag plot des donnees de $X_{t-1}$ jusqu'a $X_{t-56}$} 
	\label{fig:lagplot} 
\end{figure}

On constate graphiquement que la fonction d'autocorrélation du processus est décroissante. On peut donc affirmer
que le processus est faiblement stationnaire.

Si on pousse le lagplot un peu plus loin dans la dépendance :
\begin{figure}[H] 
	\begin{center} 
		\includegraphics[width=4in,height=4in,angle=270]{lagplot2}
	\end{center} 
	\caption{\it lag plot des données $X_{t-1}$ jusqu'à $X_{t-499}$} 
	\label{fig:lagplot2}
\end{figure}

On constate donc que la fonction fonction d'autocorrélation s'annule presque quand h augmente
. Cela nous favorise l'hypothèse que le processus est k-depandant. \\
le phénomène de périodicité est peu favorisé ici car sinon on aura pas une décroissance parfaite de 
la fonction d'autocorrélation.

\subsection{Tendance}
le processus semble ne pas avoir de saisonnalité particulière.
Le processus ne peut pas être AR ou MA car sa fonction d' autocorrélation ne s'annule pas :
\begin{figure}[H]
	\begin{center}
		\includegraphics[width=4in,height=4in,angle=270]{ac} 
	\end{center}
	\caption{}
	\label{fig:cor}
\end{figure}

