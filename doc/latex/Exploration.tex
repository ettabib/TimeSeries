\section{Exploration des données} 
    \subsection{Données}
    \subparagraph{} Les données fournies sont des cotations de spreads de CDS pour les maturités
    allant de 1ans à 10ans, entre 16/08/2005 et
    30/09/2010. 
    Cette periode a ete marque par le debut de la crise financiere en 2008, et en
    particulier par la faillite de la filiale newyorkaise de la societe AIG le 16
    septembre 2008. 
    \subparagraph{}
    Le marche des CDS est consideree comme une des raisons principales de la chute
    du groupe Americain AIG.En effet, Le march des CDS est passe de 6 396 milliards de dollars
    amricains à fin 2004 à 57 894 milliards à fin 2007, prenant le caractère d'une
    bulle financière.

    \begin{verbatim}
    > head(data)
                        cot       S1  S2  S3  S4   S5   S6   S7   S8   S9  S10
               700 2008-04-21 1.960095 7.9 8.3 9.2 10.0 10.7 11.4 12.0 12.6 13.1
               701 2008-04-22 1.791759 6.7 7.1 7.8  8.5  9.3 10.2 11.0 11.9 12.8
               702 2008-04-23 1.791759 6.7 7.1 7.8  8.5  9.3 10.1 11.0 11.8 12.7
               703 2008-04-24 1.757858 6.5 6.8 7.5  8.2  9.0  9.7 10.5 11.3 12.1
               704 2008-04-25 1.629241 5.7 6.0 6.6  7.2  8.0  8.8  9.8 10.7 11.6
               705 2008-04-28 1.648659 5.8 6.1 6.7  7.3  8.0  8.7  9.4 10.1 10.8
    \end{verbatim}

    \subsection{Chronogramme} 
    Notre première étape est donc de voir l'allure de la courbe :
    \begin{figure}[H]
        \centering
        \label{fig:chrono} 
        \includegraphics[width=3in,heigth=3in,angle=270]{chrono} 
        \caption{\it Le chronogramme des spreads de maturité 1 ans } 
    \end{figure}

    On remarque bien que la variance de la courbe augmente avec le temps. Cela
    montre de premier abord que le processus n'est pas stationnaire, autrement
    on aurait eu une variation stable au long du processus.

    On constate aussi une croissance importante du processus, et cela favorise bien
    l'existence d'une tendance croissante.

    \subsection{Stabilisation la variance}
        L'augmentation de la variance au fil du temps peu être diminue grâce a
        certaines transformation. Nous allons choisir celle  de Box-Cox définie par :
        \[
            f_\lambda(X_t)=
            \begin{cases}
                \lambda^{-1}(X_t^\lambda - 1), & X_t \geq 0,\lambda \ge 0, \\
                \ln X_t,                       & X_t \ge  0,\lambda  =   0,
            \end{cases}
        \]
        pour cela comparons les transformations pour les valeurs de
        $\lambda=0,0.5,1.5$
        \begin{figure}[H]
            \centering 
            \label{fig:Boxcox} 
            \includegraphics[width=4in,heigth=4in,angle=270]{boxcox} 
            \caption{\it Trace des spreads de 1 ans en appliquant les
            transformation de Box-Cox pour 3 valeurs de $\lambda$ } 
        \end{figure} 
        On remarque que la transformation logarithmique est bien adaptée  aux
        données car elle diminue l'effet de la croissance de la variance.\\
        Ainsi dans la suite on choisira de travailler sur cette transformation
        $\ln(X_t)$


    % \subsection{Etude de la dependance du proceccus}
        % \paragraph{Le lagplot}
        %     Un lag plot ou diagramme retardé est le diagramme de dispersion 
        %     des points ayant pour abscisse la série retardée de k instants 
        %     et pour ordonnée la série non retardée.\\ 
       
    % Les $X_i$ représentes les spreads de maturité "i" années.
    % \begin{figure}[H] 
        % \begin{center} 
        %     \includegraphics[height=4in, width=4in,angle=270]{lagplot} 
        % \end{center} 
        % \caption{\it le lag plot des données de $X_{t-1}$ jusqu'à $X_{t-56}$} 
        % \label{fig:lagplot} 
    % \end{figure}

    % On constate graphiquement que la fonction d'auto-corrélation du processus est décroissante. On peut donc affirmer
    % que le processus est faiblement stationnaire.

    % % \begin{figure}[H]
    % %     \begin{center}
    % %         \includegraphics[width=4in,height=4in,angle=0]{ac} 
    % %     \end{center}
    % %     \caption{}
    % %     \label{fig:cor}
    % % \end{figure}

    \subsection{Étude de la saisonnalité et de la tendance}

        % Dans ce paragraphe nous allons voire si on peut détecter une saisonnalité.
    % Commençons alors par tracer le périodogramme :
    % \begin{figure}[H]
    %     \centering 
    %     \label{fig:period} 
    %     \includegraphics[width=3in,heigth=3in,angle=270]{period} 
    %     \caption{\it périodogramme de la période 16/07/2007 - 03/12/2007 } 
    % \end{figure} 

    % Le périodogramme ne nous informe pas sur une probable saisonnalité.

        \subsubsection{Agregation annuelle}
            Pour avoir une idee plus claire sur le comportement annuel du
            processus nous allons tracer la moyenne des spreads par annee :
            \begin{figure}[H]
                \centering 
                \label{fig:annual} 
                \includegraphics[width=3in,heigth=3in,angle=270]{Annual} 
                \caption{\it Agregation annuelle des spreads par moyenne empirique } 
            \end{figure} 
             la societe AIG a dû
            augmenter les niveau des spreads parce que les
            entites de reference presentaient plus de risque  en periode de
            crise. 

        % \subsubsection{Agregation mensuelle}
        %     Pour voir le comportement des spreads mensuelle on va les agreger par
        %     moyenne empirique  

        %     \begin{figure}[H]
        %         \centering 
        %         \label{fig:meanMens} 
        %         \includegraphics[width=4in,heigth=4in,angle=270]{meanMens} 
        %         \caption{\it Moyenne par mois sur toutes les années pour les spreads
        %         de 1 ans } 
        %     \end{figure} 

        %     Le graphique des boxplots montre une certaine homogenite de distribution sur
        %     les mois, donc on en conclut que les données non pas une saisonnalité
        %     mensuelle particulière.

            \subsubsection{Superposition des spreads par année} 
            On va superposer les spreads
        \begin{figure}[H]
            \centering 
            \label{fig:polarAn} 
            \includegraphics[width=3in,heigth=3in,angle=270]{polarAn} 
            \caption{\it Representation polaire de les spreads } 
        \end{figure} 


        On constate bien que les points s'éloignent du centre d'année en
        année, ce qui traduit la tendance croissante du processus. En outre
        il n'y a pas de symétrie entre l'allure des spreads par année ce qui
        rend la présence d'une saisonnalité encore moins évident.

        Cette dernière observation nous dirige vers le sens d'aprouver une
        absence de saisonnalité.  

        La fonction \verb+decompose()+  en R permet de decomposer le processus en
        tendance et saisonnalite, alors que \verb+nsdiffs+ calcul le nombre de
        differenciation necessaires pour desaisonaliser le processus :
        \begin{verbatim}
        > decompose(S1)
        Erreur dans decompose(S1) : la srie temporelle a moins de 2 priodes
        \end{verbatim}

        \begin{verbatim}
        > nsdiffs(S1)
        Erreur dans nsdiffs(S1) : Non seasonal data
        \end{verbatim}


        \subparagraph{Conclusion}
        Sur ses resultats nous concluons que le processus ne presente aucune
        saisonnalité mais une tendance croissante.
           
\section{Modelisation du processus}
    \subsection{Etude des proprietes du processus}    
        \paragraph{Independance} 
        le processus est loin d'etre un processus indepandant, car certainement la
        societe n'est pas indifferente aux spreads qu'elle a emis dans le passe.
        Pour tester l'independance on choisit par exemple le test de Portemanteau
        qui est fourni par R sous \verb+Box.test()+ avec $H_0$=``le processus est
        independant''. Le test calcul
        la distance $\xi^2$ de l'auto-covariance des premiers retards. Si la p-value est tres petite ce la
        veut dire que cette auto-covariance n'est pas nulle et donc le processus :
        \begin{verbatim}
        > Box.test (S1)

        Box-Pierce test

        data:  S1
        X-squared = 1327.591, df = 1, p-value < 2.2e-16

        > Box.test (S1,  type = ``Ljung'')

        Box-Ljung test

        data:  S1
        X-squared = 1330.57, df = 1, p-value < 2.2e-16

        \end{verbatim}

        Le test retourne une p-value presque nulle donc on rejette hypothèse nulle.
        Ainsi le processus n'est pas indepandant.

        \paragraph{Stationnarite}
        Un processus stationnaire est un processus dont :
        \begin{itemize}
            \item $\forall t,E[X_t]=\mu$  
            \item $cov(X_{t+h},X_t)=\gamma(h)$ 
        \end{itemize}
        Ceci implique que le processus a une variance constante .
        Notre processus donc ne peut etre stationnaire du fait de la crise
        financière qui a eu un grand impact sur les CDS. 

        \paragraph{} Pour tester la stationnarite de notre processus nous utiliserons les tests
        de la racine unite. 
        La fonction \verb+adf.test()+ permet de faire le test de Dickey Fuller
        où l'hypothèse nulle est la non stationnarite du processus.  
        \begin{verbatim}
        > adf.test(x = S1,alternative = ``s'')

        Augmented Dickey-Fuller Test

        data:  S1
        Dickey-Fuller = -3.0033, Lag order = 11, p-value = 0.1536
        alternative hypothesis: stationary
        \end{verbatim}

        Le test a donc une p-value non negligeable donc on ne peut pas rejeter
        l'hypothese nulle de non stationnarite 
        
        \paragraph{Fonction d'auto-corrélation} tracons maintenant la fonction
        d'auto-correlation absolue et partielle :
        \begin{figure}[H]
            \centering 
            \label{fig:ac} 
            \includegraphics[width=4in,heigth=4in,angle=270]{ac} 
            \caption{\it Correlogramme } 
        \end{figure} 

        \paragraph{}l'ACF est decroissante, donc la tendance de ce processus
        pourrai bien etre une tendance polynomiale. 
        

    \subsection{Elimination de la tendance}
        \paragraph{} Les tendances polynomiale disparaissent apres un nombre de
        differenciation au minimum egale au degre polynomiale de cette tendance.
        Nous allons commencer par une differenciation d'ordre 1. Notons $Y_t=\Delta X_t$
        tracons l'allure de cette difference :

        \begin{figure}[H]
            \centering 
            \label{fig:chronoD1} 
            \includegraphics[width=3in,heigth=3in,angle=270]{chronoD1} 
            \caption{\it Allure de la difference du proceccus $X_t$ } 
        \end{figure} 

        On remarque de visu que le processus est centre. Cela peut etre confirme par
        le test de Student :
        \begin{verbatim}
        > t.test(D1,mu = 0)

            One Sample t-test

            data:  D1
            t = 1.1398, df = 637, p-value = 0.2548
            alternative hypothesis: true mean is not equal to 0
            95 percent confidence interval:
             -0.002115699  0.007969121
             sample estimates:
               mean of x 
               0.002926711 

        \end{verbatim}
        On ne peut pas rejeter donc l'hypothèse nulle qui est que le procussus
        est centre.
        \paragraph{Test d'independance}

        \begin{verbatim}
        > Box.test(x = D1)

        Box-Pierce test

        data:  D1
        X-squared = 11.272, df = 1, p-value = 0.0007869

        > Box.test(x = D1,type = ``L'')

        Box-Ljung test

        data:  D1
        X-squared = 11.2973, df = 1, p-value = 0.0007762

        \end{verbatim}
        
        La p-value du test etant tres faible on peut rejeter l'hypothèse nulle
        d'indendance : le processus depend toujours du passe.
        \paragraph{Stationnarite}
    
        En appliquant le test de Dickey-Fuller :
        \begin{verbatim}
        > adf.test(x = D1,alternative = ``s'') # -> processus stationnaire

        Augmented Dickey-Fuller Test

        data:  D1
        Dickey-Fuller = -11.0561, Lag order = 11, p-value = 0.01
        alternative hypothesis: stationary

        Message d'avis :
        In adf.test(x = D1, alternative = ``s'') :
        p-value smaller than printed p-value
        \end{verbatim}
        Le test renvoie une p-value tres faible donc on peut rejeter l'hypothèse
        nulle : le processus $Y_t$ est stationnaire.

        À present on peut envisager une modelisation ARMA pour le procussus
        $Y_t$.
    \subsection{Modelisation ARMA}        
        
        Examinons d'abord l'allure de l'auto-correlation de ce processus:
        \begin{figure}[H]
            \centering 
            \label{fig:D1ac} 
            \includegraphics[width=3in,heigth=3in,angle=270]{D1ac} 
            \caption{\it auto-correlation totale et partielle de $Y_t$ } 
        \end{figure} 

        on constate tout d'abord que l'ACF et la PACF s'annulent à partir du
        deuxieme
        retard. Donc cela nous suggère immediatement trois modeles ARMA qui peuvent
        etre candidats : ARMA(0,1), ARMA(1,0) et ARMA(1,1)
        Comparons alors ces trois modeles en utilisant le critere AIC :

        \begin{verbatim}
        > arima(s1,c(0,1,1))
        Series: x 
        ARIMA(0,1,1)                    

        Coefficients:
        ma1
        0.1231
        s.e.  0.0381

        sigma^2 estimated as 0.004142:  log likelihood=844.96
        AIC=-1687.92   AICc=-1687.9   BIC=-1679

        > arima(s1,c(1,1,0))
        Series: x 
        ARIMA(1,1,0)                    

        Coefficients:
        ar1
        0.1308
        s.e.  0.0394

        sigma^2 estimated as 0.004137:  log likelihood=845.29
        AIC=-1688.58   AICc=-1688.56   BIC=-1679.67
        > arima(s1,c(1,1,1))
        Series: x 
        ARIMA(1,1,1)                    

        Coefficients:
        ar1      ma1
        0.2877  -0.1593
        s.e.  0.2465   0.2534

        sigma^2 estimated as 0.004135:  log likelihood=845.47
        AIC=-1686.94   AICc=-1686.9   BIC=-1673.56
        \end{verbatim}

        La modelisation est clairement meilleure au niveau de la minimisation du
        critere AIC et BIC. Cela aussi est confirme par la fonction
        \verb+auto.arima()+ :
        \begin{verbatim}
        > auto.arima(x = s1,d = 1,max.p = 5,max.q = 5,start.p = 0,start.q =
        0,allowdrift = TRUE,trace = TRUE,ic = ``aic'',seasonal = FALSE)

        ARIMA(0,1,0) with drift         : -1671.466
        ARIMA(0,1,0) with drift         : -1671.466
        ARIMA(1,1,0) with drift         : -1686.102
        ARIMA(0,1,1) with drift         : -1679.346
        ARIMA(2,1,0) with drift         : -1683.538
        ARIMA(1,1,1) with drift         : -1684.192
        ARIMA(2,1,1) with drift         : -1681.549
        ARIMA(1,1,0)                    : -1686.863
        ARIMA(0,1,0)                    : -1672.167
        ARIMA(2,1,0)                    : -1684.384
        ARIMA(1,1,1)                    : -1684.962
        ARIMA(2,1,1)                    : -1682.392

        Best model: ARIMA(1,1,0)                    

        Series: s1 
        ARIMA(1,1,0)                    

        Coefficients:
        ar1
        0.1308
        s.e.  0.0394

        sigma^2 estimated as 0.004137:  log
        likelihood=845.29
        AIC=-1686.58   AICc=-1686.56   BIC=-1677.67
        \end{verbatim}
        Pour finir testons la precision du coefficient ar1 par un test
        t-statistic :
        \begin{verbatim}
                    ar1
                    t.stat 3.317992
                    p.val  0.000907
        \end{verbatim}
        la p-value est tres faible donc le coefficient est precis.
        \paragraph{Residu}
            \subparagraph{}
            \begin{verbatim}
            > t.test(M0$residuals)$p.value
            [1] 0.3169825
            \end{verbatim}
            la p-value etant tres grande on peut dire que le residu est
           centre. 
            \subparagraph{Indepandance}
                \begin{verbatim}
                > Box.test(M0$residuals)$p.value
                [1] 0.8762552
                \end{verbatim}
                La p-value du test de Portemanteau etant tres grande on peux dire
                que le residus est independant.
            \subparagraph{Stationnarite}
                \begin{verbatim}
                    > adf.test(x = M0$residuals,alternative = ``s'')$p.value
                    [1] 0.01
                    Message d'avis :
                    In adf.test(x = M0$residuals, alternative = ``s") :
                      p-value smaller than printed p-value
                \end{verbatim}
                La p-value du test de non  stationnarite est tres faible donc on
                peut rejeter l'hypothese de non stationnarite : Le residu est
                bien stationnaire
            \subparagraph{Test de Normalite} Procedons par un test de
            Shapiro et un graphe de probabilite :
            \begin{figure}[H]
                \centering 
                \label{fig:qqplotM0r} 
                \includegraphics[width=3in,heigth=3in,angle=270]{qqplotM0r} 
                \caption{\it Le graphe de probabilite du residu } 
            \end{figure} 

            Le graphe de probabilite montre clairement que le residu ne suit
            pas une loi normale
            \begin{verbatim}
            > shapiro.test(M0$residuals)$p.value
            [1] 1.011082e-24
            \end{verbatim}
            La p-value du Test de Shapiro etant tres faible donc le processus ne
            suit pas une loi normale.
            \paragraph{Conclusion} Le residu est un bruit blanc faible mais pas
            un bruit gaussien.
    \subsection{Prediction}
        \paragraph{Analyse} Pour un processus AR(1) le meilleure estimateur est l'estimateur lineaire :

        \[
            P_t(Y_{t+1}) &=&  -a_1 P_t(Y_t)
                         &=&  -a_1 Y_t  
        \]

        donc par recurrence on obtient pour $h \ge 1$
        \[
            P_t(Y_{t+h}) = (-a_1)^{h} Y_t
        \]
        Ainsi
        \[
            P_t(X_{t+h}) &=& (-a_1)^{h} Y_t + P_{t+h-1}(X_{t+h-1}) 
        \]

        \paragraph{Prediction pas a pas} On va predire le processus pas : a un t
        donne on predit t+1 a partir des vrais donnee du processus :
        \begin{figure}[H]
            \centering 
            \label{fig:pred0} 
            \includegraphics[width=3in,heigth=3in,angle=270]{pred0} 
            \caption{\it Predition pas à pas du processus $X_t$ pour les 30
            dernieres valeurs} 
        \end{figure} 

        On constate clairement l'erreur de cette prediction est tres minimale.
        \paragraph{Prediction a un plus grand pas}
        Maintenant on se base sur une seule valeur pour predire h futures
        valeurs:
        \begin{figure}[H]
            \centering 
            \label{fig:pred1} 
            \includegraphics[width=2in,heigth=2in,angle=270]{pred1} 
            \caption{\it Prediction des trois dernieres valeurs } 
        \end{figure} 
        L'estimation ici est moins bonne meme pour trois valeurs. On constate
        que les valeurs predites stagnent a partir d'un certain rang, et cela
        est dû au fait que les coefficient $(-a_1)^{h}$ ne sont plus
        significatifs au bout de 3 iterations $(h\ge3)$.
        \paragraph{Resume} Le Modele ARIMA(1,1,0) est un modele adapté a nos
        données, mais avec un bruit non gaussien. Ceci rends l'estimation des
        parametres du modele moins forte, et donc une prediction limitee.       



